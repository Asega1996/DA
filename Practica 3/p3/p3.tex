\documentclass[]{article}

\usepackage[left=2.00cm, right=2.00cm, top=2.00cm, bottom=2.00cm]{geometry}
\usepackage[spanish,es-noshorthands]{babel}
\usepackage[utf8]{inputenc} % para tildes y ñ
\usepackage{graphicx} % para las figuras
\usepackage{xcolor}
\usepackage{listings} % para el código fuente en c++

\lstdefinestyle{customc}{
  belowcaptionskip=1\baselineskip,
  breaklines=true,
  frame=single,
  xleftmargin=\parindent,
  language=C++,
  showstringspaces=false,
  basicstyle=\footnotesize\ttfamily,
  keywordstyle=\bfseries\color{green!40!black},
  commentstyle=\itshape\color{gray!40!gray},
  identifierstyle=\color{black},
  stringstyle=\color{orange},
}
\lstset{style=customc}


%opening
\title{Práctica 3. Divide y vencerás}
\author{\input{../autor}}


\begin{document}

\maketitle

%\begin{abstract}
%\end{abstract}

% Ejemplo de ecuación a trozos
%
%$f(i,j)=\left\{ 
%  \begin{array}{lcr}
%      i + j & si & i < j \\ % caso 1
%      i + 7 & si & i = 1 \\ % caso 2
%      2 & si & i \geq j     % caso 3
%  \end{array}
%\right.$

\begin{enumerate}
\item Describa las estructuras de datos utilizados en cada caso para la representación del terreno de batalla. 

La estrategia que he seguido se ha basado en la evaluación de la fila y columna
que reciben la función cellValue por parámetros y su puntuación en funcion de
su distancia al centro del mapa concreto(a mayor distancia menor puntuación)
y aumentando su puntuacion en función de su cercania a los obstaculos que puedan
existir en el ya mencionado mapa.

\begin{lstlisting}
float cellValue(int row, int col, bool** freeCells, int nCellsWidth,
               int nCellsHeight, float mapWidth, float mapHeight,
               float cellWidth, float cellHeight,
               List<Object*> obstacles, List<Defense*> defenses)
{
    float proximidad = 0;
    Vector3 dst,obj;
    dst.x = (nCellsWidth/2) * cellWidth + (cellWidth/2) - row * cellWidth +
              (cellWidth/2);
    dst.y = (nCellsHeight/2) * cellHeight + (cellHeight/2) - col * cellHeight +
            (cellHeight/2);

    for(std::list<Object*>::const_iterator it = obstacles.begin();
        it != obstacles.end();it++)
        {
        obj.x = (*it)->position.x - row * cellWidth + cellWidth/2;
        obj.y = (*it)->position.y - col * cellHeight + cellHeight/2;
        if((*it)->radio * 1.5 < obj.length())
          proximidad += 1;
        if((*it)->radio * 2 < obj.length())
          proximidad += 0.5;
        if((*it)->radio * 1.1 < obj.length())
          proximidad += 0.5;
        }
    return  std::max(mapWidth,mapHeight) - dst.length() + proximidad;
}

\end{lstlisting}


\item Implemente su propia versión del algoritmo de ordenación por fusión. Muestre a continuación el código fuente relevante. 

\begin{lstlisting}

void DEF_LIB_EXPORTED calculateAdditionalCost(float** additionalCost,
                                              int cellsWidth, int cellsHeight,
                                              float mapWidth, float mapHeight,
                                              List<Object*> obstacles,
                                              List<Defense*> defenses)
{
    Vector3 dst;  
    for(std::list<Defense*>::const_iterator def = defenses.begin();
      def != defenses.end();def++)
        for(int i = 0 ; i < cellsHeight ; ++i) {
            for(int j = 0 ; j < cellsWidth ; ++j) {
                dst.x = (*def)->position.x - (i*cellsWidth + cellsWidth);
                dst.y = (*def)->position.y - (j*cellsHeight + cellsHeight);

                additionalCost[i][j]=dst.length();
            }
        }
}


void DEF_LIB_EXPORTED calculatePath(AStarNode* originNode, AStarNode* targetNode
                   , int cellsWidth, int cellsHeight, float mapWidth, float mapHeight
                   , float** additionalCost, std::list<Vector3> &path) {

    bool nodoTerminal = false;
    float cellWidth = mapWidth/cellsWidth;
    float cellHeight = mapHeight/cellsHeight;

    AStarNode* current = originNode;

    std::vector<AStarNode*> abiertos;
    abiertos.push_back(current);
    std::vector<AStarNode*> cerrados;

    while(abiertos.size() != 0 and nodoTerminal == false){

        current = abiertos.front();
        abiertos.erase(abiertos.begin());
        cerrados.push_back(current);

        if(current == targetNode){
            nodoTerminal = true;
            path.push_front(current->position);
            current = targetNode;
        }

        else{
            int i,j;
            float distancia;
            for(List<AStarNode*>::iterator nodo = current->adjacents.begin();
                nodo != current->adjacents.end(); ++nodo)
                if(cerrados.end() == std::find(cerrados.begin(), cerrados.end(), (*nodo)))
                    if(abiertos.end() == std::find(abiertos.begin(), abiertos.end(), (*nodo))) {
                        i = (*nodo)->position.x / cellWidth;
                        j = (*nodo)->position.y / cellHeight;
                        (*nodo)->G = current->G +
                            _distance(current->position, (*nodo)->position)
                            + additionalCost[i][j];
                        (*nodo)->H =_sdistance((*nodo)->position, targetNode->position);
                        (*nodo)->F = (*nodo)->G + (*nodo)->H;
                        (*nodo)->parent = current;

                        abiertos.push_back(*nodo);
                    }
                    
                    else {
                        distancia = _distance(current->position,(*nodo)->position);

                        if((*nodo)->G > current->G + distancia) {
                            (*nodo)->G = current->G + distancia;
                            (*nodo)->F = (*nodo)->G + (*nodo)->H;
                            (*nodo)->parent = current;
                        }
                    }
                    std::sort(abiertos.begin(), abiertos.end());
        }     
    }

    while(current->parent != originNode) {
        current = current->parent;
        path.push_front(current->position);
    }
}

\end{lstlisting}



\item Implemente su propia versión del algoritmo de ordenación rápida. Muestre a continuación el código fuente relevante. 

\begin{lstlisting}
void algoritmo_mochila(unsigned int ases, std::list<Defense*> defenses,
                       float rendimiento[], unsigned int coste[],float** matDef)
{

  for(int i=0;i<=ases;i++){
    if(i<coste[0])
    {
      matDef[0][i]=0;
    }
    else
    {
      matDef[0][i]=rendimiento[0];
    }
  }

  for(int i=1;i<defenses.size();i++){
  for(int j=0;j<=ases;j++)
    if(j<coste[i])
      matDef[i][j]=matDef[i-1][j];

    else
      matDef[i][j]=std::max(matDef[i-1][j], matDef[i-1][j-coste[i]] +
                                            rendimiento[i]);
  }

}
\end{lstlisting}


\item Realice pruebas de caja negra para asegurar el correcto funcionamiento de los algoritmos de ordenación implementados en los ejercicios anteriores. Detalle a continuación el código relevante.

Pruebas cajas negras para Fusion:
\begin{lstlisting}
//Codigo previo
ordenacionFusion(v);
//Codigo previo
for(int i = 0;i != v.size() - 1; i++ ){
  if(v[i].puntuacion > v[i+1].puntuacion)
    cajaNegraFusion = true; //Estará a true cuando falle el algoritmo
\end{lstlisting}

Pruebas caja negra Ordenacion Rapida:
\begin{lstlisting}
//Codigo previo
quicksort(v,0,v.size()-1);
//Codigo previo
for(int i = 0;i != v.size() - 1; i++){
  if(v[i].puntuacion > v[i+1].puntuacion)
    cajaNegraRapida = true; //Estará a true cuando falle el algoritmo
}
\end{lstlisting}

Pruebas caja negra Monticulo:
\begin{lstlisting}
std::make_heap(v.begin(),v.end());
std::sort_heap(v.begin(),v.end());
for (size_t i = 0; i < v.size()-1; i++) {
  if(v[i].puntuacion > v[i+1].puntuacion)
    cajaNegraMonticulo= true; //Estaráa true cuando falle el algoritmo
}
\end{lstlisting}


\item Analice de forma teórica la complejidad de las diferentes versiones del algoritmo de colocación de defensas en función de la estructura de representación del terreno de batalla elegida. Comente a continuación los resultados. Suponga un terreno de batalla cuadrado en todos los casos. 

En esta función he intentado puntuar con mayor valor las celdas mas cercanas en
linea recta al centro de extracción ya colocado, de manera que estaría anillando
las posiciones cercanas al centro, tratando de conseguir que las defensas rodeen
el ya mencionado centro.

\begin{lstlisting}
void cellValueDefensas(float** cellValue, int  nCellsWidth, int nCellsHeight,
              float cellWidth, float cellHeight,float mapWidth, float mapHeight,
              std::list<Defense*> defenses)
{
    List<Defense*>::const_iterator centroExtraccion = defenses.begin();
    Vector3 dst;
    for(int i = 0; i < nCellsWidth; ++i)
        for(int j = 0; j < nCellsHeight; ++j)
        {
          dst.x = i*cellWidth + cellWidth/2 - (*centroExtraccion)->position.x;
          dst.y = j*cellHeight + cellHeight/2 - (*centroExtraccion)->position.y;
          cellValue[i][j] = std::max(mapWidth, mapHeight) - dst.length();
        }
}
\end{lstlisting}


\item Incluya a continuación una gráfica con los resultados obtenidos. Utilice un esquema indirecto de medida (considere un error absoluto de valor 0.01 y un error relativo de valor 0.001). Considere en su análisis los planetas con códigos 1500, 2500, 3500,..., 10500. Incluya en el análisis los planetas que considere oportunos para mostrar información relevante.

\begin{lstlisting}
void DEF_LIB_EXPORTED placeDefenses(bool** freeCells, int nCellsWidth,
                                    int nCellsHeight, float mapWidth,
                                    float mapHeight,
                                    std::list<Object*> obstacles,
                                    std::list<Defense*> defenses)
{
    //Continuacion directa del codigo del ejercicio 3.

    maxAttemps = 1000 * std::max(nCellsWidth,nCellsHeight);

    while(currentDefense != defenses.end() && maxAttemps > 0)
    {
        seleccion(cellValues, nCellsWidth, nCellsHeight, &fila, &columna);

        x = fila*cellWidth + cellWidth/2;
        y = columna*cellHeight + cellHeight/2;

        if(factible(x, y, (*currentDefense)->radio, mapWidth, mapHeight,
        cellWidth,cellHeight, obstacles, defenses))
        {
            (*currentDefense)->position.x = x;
            (*currentDefense)->position.y = y;
            (*currentDefense)->position.z = 0;
            currentDefense++;
        }
        maxAttemps--;
    }

  }
\end{lstlisting}


\end{enumerate}

Todo el material incluido en esta memoria y en los ficheros asociados es de mi autoría o ha sido facilitado por los profesores de la asignatura. Haciendo entrega de este documento confirmo que he leído la normativa de la asignatura, incluido el punto que respecta al uso de material no original.

\end{document}
