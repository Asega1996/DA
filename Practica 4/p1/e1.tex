La estrategia que he seguido se ha basado en la evaluación de la fila y columna
que reciben la función cellValue por parámetros y su puntuación en funcion de
su distancia al centro del mapa concreto(a mayor distancia menor puntuación)
y aumentando su puntuacion en función de su cercania a los obstaculos que puedan
existir en el ya mencionado mapa.

\begin{lstlisting}
float cellValue(int row, int col, bool** freeCells, int nCellsWidth,
               int nCellsHeight, float mapWidth, float mapHeight,
               float cellWidth, float cellHeight,
               List<Object*> obstacles, List<Defense*> defenses)
{
    float proximidad = 0;
    Vector3 dst,obj;
    dst.x = (nCellsWidth/2) * cellWidth + (cellWidth/2) - row * cellWidth +
              (cellWidth/2);
    dst.y = (nCellsHeight/2) * cellHeight + (cellHeight/2) - col * cellHeight +
            (cellHeight/2);

    for(std::list<Object*>::const_iterator it = obstacles.begin();
        it != obstacles.end();it++)
        {
        obj.x = (*it)->position.x - row * cellWidth + cellWidth/2;
        obj.y = (*it)->position.y - col * cellHeight + cellHeight/2;
        if((*it)->radio * 1.5 < obj.length())
          proximidad += 1;
        if((*it)->radio * 2 < obj.length())
          proximidad += 0.5;
        if((*it)->radio * 1.1 < obj.length())
          proximidad += 0.5;
        }
    return  std::max(mapWidth,mapHeight) - dst.length() + proximidad;
}

\end{lstlisting}
